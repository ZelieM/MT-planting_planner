% Background
% In  this  section,  introduce  the  background  material  needed 
% for the reader to understand the core chapters of your thesis. 
% This includes all relevant material to make the thesis “self-
% contained”  so  that  reader  can  understand  the  entire  thesis 
% without having to consult other sources. For example if you 
% use a certain algorithm, or if you target a certain platform, 
% or if you rely on a certain mathematical theory, it may be 
% useful to summary the essence of that material here, so that 
% the  user  can  fall  back  on  it  if  needed.  Users  that  already 
% know this material should be able to safely skip this section. 
% Do not forget to explain clearly to the reader why this ma-
% terial  is  relevant  in  the  context  of  this  thesis,  so  that  he 
% doesn’t get bored by all this extra stuff, which is only back-
% ground  material  but  not  part  of  the  essence  of  the  thesis. 
% Don’t exaggerate with this section. Only put what is really 
% necessary  to  understand  what  comes  later.  Some  theses 
% don’t need a background section at all. 


% Related work
% In this section you should list relevant related work existing 
% today:  what  other  solutions  are  provided  today  to  address 
% the problem that you stated in the introduction. You could: 
% provide  a  summary  of  some  articles  that  you  have  read, 
% make a comparison table between solutions that you found 
% relevant for the problem, analyze existing solutions, discuss 
% advantages  and  shortcomings  of  some  selected  solutions, 
% discuss  in  details  the  limitations  of  existing  practice.  The  
% main point of the related work is to put in perspective your 
% work,  which  will  build  upon  or  be  complementary  to  that 
% related work. 

% \begin{itemize}
%     \item En informatique/ software engineering : Django, Python, Agile, Continuous Integration, Code coverage
%     \item En agro: qu'est ce qui existe dans le domain. Bases de données, recherches,... Métier de maraicher, ce qu'ils font au quotidien, l'organisation d'une année-type
% \end{itemize}







